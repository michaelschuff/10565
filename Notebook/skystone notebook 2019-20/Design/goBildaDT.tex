\begin{document}

\subsection{goBilda Drivetrain}
\par The second drivetrain that was CADed was a drivetrain that, excluding mecanum wheels, was used of all goBilda structure and motion. The goBilda building system is very modular and more customizable than any other, and because of this, there was the possibility of using it for the complete robot.

\img{Images/gobildachassis.png}{}{.9}

\wrap{r}{Images/cantileverwheel.png}{.4}

\par \textbf{Cantilever Support:} One of the main differences of this robot from the previously CADed one is the way the mecanum wheels are supported. In this design, the axles are cantilevered, meaning that they are supported on one side of the wheel, but still in two places. Here we have two goBilda half-channel pieces that have ball-bearing pillow blocks to support the axle. Although it is sturdier to support it on both sides of the wheel, the drivetrain uses 11mm hex axles, which are more than enough to support the robot. One detriment of having a cantilever drivetrain is that there is no protection of the wheels. We would need to add some plates on the outside to put our number and help protect from the possible impact with the field or other robots.

\par \textbf{Custom Parts are Negligible:} One big benefit of the goBilda drivetrain is that there is no need to machine any parts, since all of it comes in the correct state. The only custom part would be the belt pulleys, which would easily be 3d-printed. Once the parts arrive, we would be able to complete the chassis in a few hours, which is extremely beneficial. This also means that troubleshooting the robot is a lot easier. If something breaks, then we could just replace it compared to having to make a new one. In previous years, the first one and a half months of robotics practice would be used to make the drivetrain, and with this one, that would be done in one practice, leaving lots of room for focusing on other mechanisms and tasks.
\newpage

\wrap{l}{Images/gobildapattern.jpg}{.6}

\par \textbf{goBilda Pattern:} Using goBilda means that we have easy access to use the goBilda building system. One benefit is the pattern that all the goBilda parts use. This pattern makes building very easy since there are many mounting points. It isn't as accessible as REV, but it is the next best thing. All goBilda parts have this pattern in mind, and anything goBilda will be able to fit with each other. Using the goBilda half-channel for plates means that we can attach mechanisms to both the sides of the robot and the half-channel that connects the two sides. The goBilda also makes it so that our robot can be modular and we can just attach mechanisms to it as we please. This makes prototyping very easy. If there is anything that needs to be adjusted or changed, we wouldn't have to change the whole robot or drivetrain, we would just need to take the mechanism off and change it.

\img{Images/goBildaSpreadsheet}{goBilda parts spreadsheet to calculate cost. The "After Discount" column is using the goBilda FTC discount.}{1}
\proConTab{this \par is}{a \par test}
\newpage

\end{document}