\begin{document}

\subsection{goBilda Drivetrain Finalized}

\img{Images/goBildaNewFullRobot}{}{.9}

\par After the first meeting with the team, the three CADed drivetrains were discussed, and we decided on making a goBilda drivetrain for the beginning of the season. We also discussed the possibility of buying a goBilda FTC kit. This kit would help with the first stages of designing and prototyping a robot. Because of this possibility, we wanted to make a drivetrain that would be able to use the parts that come with the kit.

\wrap{l}{Images/goBildaNewWheel}{.6}

\par \textbf{Full Channel:} The first difference from the original goBilda drivetrain are the sides. The first one was made of four pieces of half-channel, compared to this one which uses two pieces of regular channel. This channel means that the sides of the robots would be wider, which makes it so that there is less space in the middle of the robot. Because it uses the channel, the robot will have the full goBilda pattern on the top of the drivetrain. This is better than the original goBilda drivetrain which had two lines of holes on the top of the half-channel because there are many more attachment points.

\newpage

\wrap{r}{Images/gobildapillowblocks}{.45}

\par \textbf{Custom Pillow Blocks:} In order to reduce cost, this drivetrain uses 3d-printed pillow blocks for the ball bearings. These are used instead of the ones that are available from goBilda. These are used to hold the ball bearings that go on the axles for the mecanum wheels. These pillow blocks don't necessarily hold in the ball bearings, so spacers are needed to keep everything in between the blocks to be rigid in their position.


\par \textbf{Motor Mounts:} Right now, we have Rev 20:1 orbital motors, and we don't want to buy new goBilda ones. Since the motors are Rev, they have a different pattern on the face of the motor than goBilda. This means that we can't just attach the motors onto the side of the channel, so we designed some motor mounts. These have the right measurements so that we can attach the Rev motors onto the mounts, and there are also holes to attach the mounts to the goBilda. There are also sunken holes where the screw head is so that they don't interfere with the channel face. One mount instead of two is better because it adds structural durability instead of two separate pieces.
\wrap{l}{Images/goBildaNewMotorMount}{.45}

%insert paragraph for the stress testing and whatever

\par \textbf{Cost:} Overall, this drivetrain will cost \$76.89. This is the least expensive drivetrain that has been made, and still uses mostly goBilda parts. The box tube drivetrain is the second cheapest, and that uses a lot of custom parts, like the box tube compared to this one that still uses goBilda channel for the sides. A lot of the cost from the other goBilda drivetrain is due to the half-channel and ball bearing pillow blocks, which both go down with this drivetrain.

\img{Images/newgobildaspreadsheet}{}{1}

\newpage

\end{document}