\begin{document}

\subsection{Box Tube Drivetrain}
\par The third drivetrain that was designed over the summer uses box tube to support and attach the wheels to the chassis. It is also cantilever, and aside from the box tube, it is very similar to the goBilda one.

\img{Images/boxTubeDT.png}{}{.9}

\wrap{r}{Images/boxtube}{.65}

\par \textbf{Box Tube:} Using box tube comes with both benefits and detriments. One benefit is that it is a very common industrial part. It is two inches by one inch, and is 1/8 of an inch thick. It is made of aluminum, and at that thickness, it is very sturdy. Using a custom structure means that we can use other industrial grade materials that can't be used well with a building structure like goBilda. With the box tube, it gives us access to use very cheap, yet reliable ball bearings. Since the box tube is an inch wide, it means that the sides of the robot are only three inches thick, thinner than the previous two drivetrains. Having it be thinner means that there is more room for mechanisms on the inside of the robot. Using box tube also has some issues. The first is that since the box tube is a box, it is hard to access the inside, which is necessary to put the pulleys inside. To solve this issue, we would need to use a CNC mill to pocket the box tube for easy access to the inside of it. This would require time to do, and would have to wait until we can get anything else done. Not having easy access to the inside also means that it would be harder to build the drivetrain as a whole and harder to troubleshoot if something breaks.

\newpage

\wrap{l}{Images/spacer}{.5}

\par \textbf{Customizability:} Having the box tube means that the drivetrain is a lot easier to customize to fit requirements for other mechanisms. Although it isn't as good as using goBilda, it is very easy to drill into the box tube to attach parts. Both the custom and goBilda drivetrains have good accessibility on the sides, since they use plates, but neither of them have as much access to the top and bottom. Using box tube means that building off the top is much easier. On both the goBilda and custom drivetrains, there is only one distance between the mounting points, while with box tube you can mount something all along the one inch space. Easy access to drilling holes in the box tube means that we could use goBilda half-channel to connect the two sides of the drivetrain. This makes it even easier to attach things to the robot. The only issue is that the box tube is imperial and goBilda is metric. Since the half-channel is 48mm in height, it is just a little bit smaller than the height of the box tube. To fix this, we made a custom spacer to fill in that gap and make it so the two parts can be attached by some triangle brackets.

\par \textbf{Simplicity:} Although machining the box tube will be a little hard to do, the rest of the drivetrain is very simplistic. The creation of the drivetrain is very simple, with the wheels only having an axle, two ball bearings, a pulley, and a wheel. Then there are just two pieces of goBilda half-channel attached to the box tube by two triangle brackets each. The motors are attached directly onto the box tube and there are some hubs that attach the belt pulleys to the motor. Excluding the wheels and motors, the total cost of the drivetrain would be \$77.73. This is more than half the cost of the other two drivetrains. 
\\

\img{Images/boxTubeSpreadsheet.png}{Box tube drivetrain cost of parts spreadsheet.}{1}

\proConTab{this \par is}{a \par test}

\end{document}