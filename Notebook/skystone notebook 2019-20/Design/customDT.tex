\begin{document}

\subsection{Custom Drivetrain}
\par The first drivetrain that we designed over the summer was a mostly custom parallel-plate mecanum drive.

\wrap{r}{Images/sideplate.png}{.5}

\par \textbf{Aluminum Side Plates:} This first drivetrain uses aluminum plates on each side of the wheels. These plates support the wheels so that the weight of the robot is dispersed evenly upon the axles. The side plates are also used as protection for the wheels, so that any impact onto the robot will not damage any key components to the drivetrain. The side plates have a separation of almost 3 inches and are pocketed to remove unnecessary aluminum to reduce weight but still maintain it's structural integrity. Since there is pocketing and the side plates require precise cutting and drilling into aluminum, a CNC mill would be required. Using the mill would mean that would be an extra step in the building process and would need access to one. Our school's metal shop has one, but that means we would need to find time to use it during school hours.

\wrap{l}{Images/customDT.jpg}{.63}

\par \textbf{Dead Axles:} The axles that support the wheels to the actual chassis are not driven by the driving motors. The belt pulleys that are connected to the motors are directly attached to the wheel, so there is no motion that goes through the axle. This adds to the structurally integrity of the drivetrain because it allows the axles to be bolted to the side plates. The dead axles also make it so no torque is applied through them, so there is no possibility of any twisting.

\wrap{1}{Images/motorplate.png}{.4}

\par \textbf{Motor Attachment:} Another feature of the custom drivetrain is how the motors are attached to it. The two motors that are used to drive one side of the drivetrain are attached to their own plate, separate from the inner side plate. This plate is then attached to the outer side plate, using the empty space on the inside of the two plates. This is beneficial because it limits the amount of empty space, meaning the motors don't take up as much on the inside of the robot, which leaves more room for mechanisms.

\par \textbf{Modulability:} The two sides of the drivetrain are connected by two cuts of peanut extrusion to support and build off of. Peanut extrusion has holes on both ends that are threaded, so bolts can be screwed into them. This means that there is no need to use gussets or other forms of support to attach the two sides of the drivetrain. The peanut extrusion is extremely sturdy, so there would be no need to worry about the drivetrain break. One issue with this custom design is that there isn't that much room for prototyping and testing because we would need to drill holes to attach mechanisms to it. One way that this is countered in this drivetrain is that there are cuts of REV extrusion on top of the side plates, so there is more room for modulability. Another way that we helped dissolve this issue was to add more holes to the sides of the inner side plates. They aren't necessarily in any pattern to attach mechanisms, but they are used to move around the peanut extrusion if needed. These changes don't get rid of the issue, but they help. A custom drivetrain would be the best for a robot which is fully designed and CADed, not for prototyping.
\\
\img{Images/customSpreadsheet.png}{Custom drivetrain cost and parts spreadsheet}{1}
\newpage
\proConTab{this \par is}{a \par test}
\newpage
\end{document}